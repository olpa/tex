% \iffalse meta-comment
%
% Copyright (C) 2010-2014 by Oleg Parashchenko <olpa@uucode.com>
% -------------------------------------------------------
% 
% This file may be distributed and/or modified under the
% conditions of the LaTeX Project Public License, either version 1.2
% of this license or (at your option) any later version.
% The latest version of this license is in:
%
%    http://www.latex-project.org/lppl.txt
%
% and version 1.2 or later is part of all distributions of LaTeX 
% version 1999/12/01 or later.
%
% \fi
%
% \iffalse
%<*driver>
\ProvidesFile{cals.dtx}
%</driver>
%<package>\NeedsTeXFormat{LaTeX2e}[1999/12/01]
%<package>\ProvidesPackage{cals}
%<*package>
    [2015/11/03 CALS tables v.2.3.2]
%</package>
%
%<*driver>
\documentclass{ltxdoc}
\usepackage{hyperref}
\EnableCrossrefs
\CodelineIndex
\RecordChanges
\begin{document}
  \DocInput{cals.dtx}
  \PrintChanges
  \PrintIndex
\end{document}
%</driver>
% \fi
%
% \CheckSum{0}
%
% \CharacterTable
%  {Upper-case    \A\B\C\D\E\F\G\H\I\J\K\L\M\N\O\P\Q\R\S\T\U\V\W\X\Y\Z
%   Lower-case    \a\b\c\d\e\f\g\h\i\j\k\l\m\n\o\p\q\r\s\t\u\v\w\x\y\z
%   Digits        \0\1\2\3\4\5\6\7\8\9
%   Exclamation   \!     Double quote  \"     Hash (number) \#
%   Dollar        \$     Percent       \%     Ampersand     \&
%   Acute accent  \'     Left paren    \(     Right paren   \)
%   Asterisk      \*     Plus          \+     Comma         \,
%   Minus         \-     Point         \.     Solidus       \/
%   Colon         \:     Semicolon     \;     Less than     \<
%   Equals        \=     Greater than  \>     Question mark \?
%   Commercial at \@     Left bracket  \[     Backslash     \\
%   Right bracket \]     Circumflex    \^     Underscore    \_
%   Grave accent  \`     Left brace    \{     Vertical bar  \|
%   Right brace   \}     Tilde         \~}
%
%
% \changes{v2.4}{2014/11/12}{Built-in RTL (right-to-left) support.}
% \changes{v2.2}{2013/05/27}{Hooks for bidi support.}
% \changes{v2.1}{2013/05/24}{Alignment of tables works.}
% \changes{v2.0}{2010/10/08}{Complete rewrite with different approach}
% \changes{v1.0}{2010/06/17}{Initial version from skeleton.dtx}
%
% \GetFileInfo{cals.dtx}
%
% \DoNotIndex{\newcommand,\newenvironment}
% 
%
% \title{The \textsf{cals} package\thanks{This document
%   corresponds to \textsf{cals}~\fileversion, dated \filedate.}}
% \author{Oleg Parashchenko \\ \texttt{olpa@uucode.com}}
%
% \maketitle
%
% \section{Introduction}
%
% The |cals| package is a set of macros to typeset multipage tables
% with repeatable headers and footers, with cells spanned over
% rows and columns. Decorations are supported: padding, background
% color, width of separation rules. The code is compatible with
% multicols and bidi.
%
% The work is released to public (\LaTeX{} license)
% by \url{bitplant.de}~GmbH, a company which provides technical
% documentation services to industry.
%
% \section{Usage}
%
% The users' guide is a separate document, published in TUGboat 2011:2:
% \url{http://tug.org/TUGboat/tb32-2/tb101parashchenko.pdf}
%
% The most important feature: the table (its rows) must start in a
% vertical mode, the cells content should switch to a horizontal mode.
%
% Please post questions and suggestions to TeX-SX
% (\url{http://tex.stackexchange.com/}), the newsgroup
% |comp.text.tex| and the |texhax| mailing list (see
% \url{http://tug.org/mailman/listinfo/texhax}),
% not directly to me.
%
% Summary of the user interface:
% \begin{verbatim}
% \begin{calstable}
% \colwidths{{100pt}{200pt}}
% \brow \cell{a} \cell{b} \erow
% \end{calstable}
% \end{verbatim}
%
% Table elements: |\thead|, |\tfoot|, |\tbreak{\penalty-10000}|, |\lastrule|.
%
% Alignment: |\alignL|, |\alignC|, |\alignR|, |\vfill|.
%
% Padding: lengths |\cals@paddingL| (|...T,R,B|), set by |\cals@setpadding{Ag}|,
% baseline alignment |\cals@paddingD|, set by |\cals@setcellprevdepth{Al}|.
%
% Color: |\cals@bgcolor|.
%
% Rules: |\cals@cs@width|, |\cals@framecs@width|, |\cals@rs@width|,
% |\cals@framers@width|, |\cals@bodyrs@width|. Overrides:
% |\cals@borderL| (|...T,R,B|).
%
% Hooks: |\cals@AtBeginTable|, |\cals@AtEndTable|,
% |\cals@AtBeginCell|, |\cals@AtEndCell|.
%
% Spanning: |\nullcell|, |\spancontent|.
%
% \StopEventually{}
%
% \section{Implementation}
%
% What happens. |\cell| creates a table cell, puts it to the
% current row and updates decorations. At the end of the row
% (|\erow|) we have the box |\cals@current@row|, the box
% |\cals@current@cs|---column separation and cells background---and
% the macros |\cals@current@rs@above| and |\cals@current@rs@below|---all
% the required data to typeset row separation.
% Before dispatching the row, all the cells are repacked
% to the common height. The row dispatcher (|\cals@row@dispatch|)
% usually just uses |\cals@issue@row|, which
% outputs |current@cs|, then joins the previous row |cs@below| with the
% current row |rs@above| and typesets the resulting row separation, and
% finally prints the row itself. If a table break is required, the
% dispatcher backups the current row and first typesets the table footer,
% a page break and the table header. In case of a row span, the set
% of the rows is converted to one big row.
%
% I tried to code as good and robust as I can. In particular,
% the package contains unit tests. However, being an unexperienced
% \TeX{} programmer, I could write bad code, especially in the section
% ``List list of tokens''. Do not hesitate to send me suggestions
% and corrections, also in the use of English.
%
% The description is split on two parts: main functionality and
% decorations. The first part is bottom-up: creating cells,
% collecting cells to a row, dispatching a row, top-level table
% elements. The second part starts with the common code,
% then explains in-row decorations (column separation and cells
% background) and between-row decorations (row separation).

%
% % \subsection{Creating cells}

% \begin{macro}{\cals@cell}
% \begin{macro}{\cals@cell@end}
% Creates an individual cell before socialization into a table.
% Content of the cell is typeset inside a group.
% Execution continues in |\cals@celll@end|. Parameters:
% \begin{enumerate}
% \item Width of the cell
% \item Vertical correction: when we have a rowspan, the cell is created while processing the last row. The vertical correction is required to raise the text back to the first row of the rowspan.
% \item (Implicit parameter.) Content. It is important that it contains a switch to the horizontal mode, otherwise horizontal dimensions of the cell will be incorrect.
% \end{enumerate}
% Using an implicit parameter instead of putting it to a macro
% parameter is probably a premature optimization.
%
%
%    \begin{macrocode}
\newcommand\cals@cell[3]{}
\def\cals@cell#1#2{%
%    \end{macrocode}
%
% Start immediately with |\vbox| to allow |\setbox0=\cals@cell{...}|
% construction. Later, white integrating the cell into
% a row, the content will be unvboxed and put to a vbox
% of the row height.
%
%    \begin{macrocode}
\vbox\bgroup%
%    \end{macrocode}
%
% Implicitely sets the width and the horizontal paddings
% of the cell. These settings come into effect on
% switch from the restricted vertical mode (our |\vbox|) to
% the horizontal mode. Therefore, the content must force such
% switch, otherwise the code fails.
%
%    \begin{macrocode}
\hsize=#1
\linewidth=#1
\leftskip=\cals@paddingL %
\rightskip=\cals@paddingR %
%    \end{macrocode}
%
% Vertical correction and top padding
%
%    \begin{macrocode}
\ifdim #2>0pt %
 \vskip-#2
\fi
\vskip\cals@paddingT %
%    \end{macrocode}
%
% Tuning the top padding. First, compensate the |\parskip|,
% which appears on the mode switch. Second, adjusts baselineskip,
% so in the case of the right preliminimary setup, the top of the
% letters "Al" touches the padding border. Meanwhile, setting
% prevdepth aligns the baselines of the first text lines
% of the row cells.
%
%    \begin{macrocode}
\vskip-\parskip %
\prevdepth=\cals@paddingD %
%    \end{macrocode}
%
% Finally, the content. And the switch to the horizontal mode (we hope).
%
% We want more work afrer typesetting the content, but it is not
% desirable to collect all the tokens. Instead, start a group and use
% |\aftergroup| to finish typesetting. For more explanations, see
% ``\TeX{} by Topic'', Chapter 12 ``Expansion''.
%
%    \begin{macrocode}
\bgroup\aftergroup\cals@cell@end
\cals@AtBeginCell\let\next=% eat '{' of the content
}%{Implicit content}

%    \end{macrocode}
%
% The infinite glue before the bottom padding is useful later,
% when we will re-height the cells in a row.
%
%    \begin{macrocode}
\def\cals@cell@end{\vfil\vskip\cals@paddingB
\cals@AtEndCell\egroup % finish vbox
%    \end{macrocode}
% Call the caller
%    \begin{macrocode}
\cals@celll@end}
%    \end{macrocode}
% \end{macro}
% \end{macro}

% \begin{macro}{\cell}
% Creates a cell and appends it to the hbox |\cals@current@row|.
%    \begin{macrocode}
\newcommand\cell[1]{}
\def\cell{%
%    \end{macrocode}
% Get the width of the cell and typeset it to the box 0.
% The execution flow is: |\cell| to |\cals@cell|
% to |\cals@cell@end| to |\cals@celll@end|.
%    \begin{macrocode}
\llt@rot\cals@colwidths
\let\cals@cell@width=\llt@car
\setbox0=\cals@cell\cals@cell@width{0pt}%
}
%    \end{macrocode}
% \end{macro}

% \begin{macro}{\cals@AtBeginCell}
% \begin{macro}{\cals@AtEndCell}
% Additional code to be executed at the begin and at the end
% of a cell. An use case is a hook for pdfsync:
% |\def\cals@AtBeginCell{\pdfsyncstart}|. I am not sure if the end-hook
% is useful because all the changes are local for the cell group,
% but decided to retain it for symmetry.
%    \begin{macrocode}
\let\cals@AtBeginCell=\relax
\let\cals@AtEndCell=\relax
%    \end{macrocode}
% \end{macro}
% \end{macro}

% \begin{macro}{\cals@width@cell@put@row}
% Implicit setting of the cell width can fail (example is an empty cell).
% In this case, force the width explicitely. Then put the cell
% to the current row. This code should be a part of |\cals@celll@end|,
% but due to |\spancontent| is in a separate macro.
%    \begin{macrocode}
\newcommand\cals@width@cell@put@row{%
\ifdim \cals@cell@width=\wd0 \relax \else \wd0=\cals@cell@width \fi
\setbox\cals@current@row=\hbox{\unhbox\cals@current@row\box0 }}%
%    \end{macrocode}
% \end{macro}

% \begin{macro}{\cals@celll@end}
% After a cell is typeset to the box 0, execution continues here
% (see notes to |\cell|). Update the current row and its decorations.
%    \begin{macrocode}
\newcommand\cals@celll@end{%
\cals@width@cell@put@row
\cals@decor@next\cals@cell@width}
%    \end{macrocode}
% \end{macro}

% \begin{macro}{\spancontent}
% Typesets a spanned cell (the content is in the implicit argument)
% and puts it to the current row.
% The width and height correction are
% already calculated, the decorations are also already added.
%    \begin{macrocode}
\newcommand\spancontent[1]{}
\def\spancontent{%
\let\cals@tmp=\cals@celll@end
\let\cals@cell@width=\cals@span@width
\def\cals@celll@end{%
  \cals@width@cell@put@row%
  \let\cals@celll@end=\cals@tmp}%
\setbox0=\cals@cell{\cals@span@width}{\cals@span@height}%
}%{Implicit content}
%    \end{macrocode}
% \end{macro}

% \subsubsection{Cell padding}

% \begin{macro}{\cals@setpadding}
% \begin{macro}{\cals@paddingL}
% \begin{macro}{\cals@paddingR}
% \begin{macro}{\cals@paddingT}
% \begin{macro}{\cals@paddingB}
% 
% Calculates and sets the cell padding. It seems that a good value
% is the half of the font size, calculated as the full height of
% a box with the content \#1. The calstable environment uses
% the letters ``Ag''.
%
%    \begin{macrocode}
\newskip\cals@paddingL
\newskip\cals@paddingR
\newskip\cals@paddingT
\newskip\cals@paddingB

\newcommand{\cals@setpadding}[1]{%
\setbox0=\hbox{#1}%
\dimen0=\ht0 \advance\dimen0 by \dp0 \divide\dimen0 by 2
\cals@paddingL=\dimen0 \relax
\cals@paddingR=\cals@paddingL
\cals@paddingT=\cals@paddingL
\cals@paddingB=\cals@paddingL
}
%    \end{macrocode}
% \end{macro}\end{macro}\end{macro}\end{macro}\end{macro}

% \begin{macro}{\cals@setcellprevdepth}
% \begin{macro}{\cals@paddingD}
% The function |\cals@cell| uses the
% length |\cals@paddingD| to tune the top padding. The macro
% |\cals@setcellprevdepth| calculates and sets this parameter,
% so that a box with the content \#1 touches the padding border.
% The calstable environment uses the letters ``Al''.
%
%    \begin{macrocode}
\newdimen\cals@paddingD

\newcommand{\cals@setcellprevdepth}[1]{%
\setbox0=\vbox{\prevdepth=0pt #1}%
\setbox1=\vbox{#1}%
\dimen0=\ht0 \advance\dimen0 by \dp0 %
\advance\dimen0 by -\ht1 \advance\dimen0 by -\dp1%
\cals@paddingD=\dimen0 }
%    \end{macrocode}
% \end{macro}\end{macro}

% \begin{macro}{\alignL}
% \begin{macro}{\alignC}
% \begin{macro}{\alignR}
% To align the table cell text left, center or right we add or
% remove vfill-part of the left and right padding. Executed
% by assigning skip to dimen.
%    \begin{macrocode}
\newcommand\alignL{%
\dimen0=\cals@paddingL \cals@paddingL=\dimen0 \relax
\dimen0=\cals@paddingR \cals@paddingR=\dimen0 \relax}

\newcommand\alignC{%
\dimen0=\cals@paddingL \cals@paddingL=\dimen0 plus 1fill\relax
\dimen0=\cals@paddingR \cals@paddingR=\dimen0 plus 1fill\relax}

\newcommand\alignR{%
\dimen0=\cals@paddingL \cals@paddingL=\dimen0 plus 1fill\relax
\dimen0=\cals@paddingR \cals@paddingR=\dimen0 \relax}
%    \end{macrocode}
% \end{macro}
% \end{macro}
% \end{macro}

% \subsection{From cells to a row}

% \begin{macro}{\cals@current@row}
% Rows are first typeset to this hbox.
%    \begin{macrocode}
\newbox\cals@current@row
%    \end{macrocode}
% \end{macro}

% \begin{macro}{\colwidths}
% \begin{macro}{\cals@colwidths}
% The macro |\cals@colwidths| contains a list of column widths.
% The user sets it through the API macro |\colwidths|, which performs
% expansion. The list is alive, it is rotated after a cell is finished,
% so the width of the next cell is always the first element.
%    \begin{macrocode}
\newcommand\colwidths[1]{}
\def\colwidths#{\edef\cals@colwidths}
\def\cals@colwidths{{100pt}}
%    \end{macrocode}
% \end{macro}\end{macro}

% Initially, I planned to use |\row{...}| to create a row.
% In order to perform actions at the end of the row I used
% the aftergroup-trick, like for |\cell| command. Unfortunately,
% the decorations were lost after the
% group finished. I tried to save them to global temporary
% macros at the end of each cell, but the saving list was big.
% Finally, I decided that the construction |\brow...\erow|
% is much easier to implement.

% \begin{macro}{\brow}
% Starts a row. Resets the rowspan markers, |\cals@current@row|
% and decorations.
%    \begin{macrocode}
\newcommand\brow{%
\cals@updateRspanMarkers
\setbox\cals@current@row=\hbox{}%
\cals@decor@begin}
%    \end{macrocode}
% \end{macro}

% \begin{macro}{\erow}
% Finishes a row. All the cells are re-layouted to the row height.
% Decorations are finalized, and the row is dispatched.
%    \begin{macrocode}
\newcommand\erow{%
\cals@reheight@cells\cals@current@row
\cals@last@row@height=\ht\cals@current@row\relax
\cals@decor@end\cals@lastWidth
\ht\cals@current@cs=\ht\cals@current@row
\cals@row@dispatch
}
%    \end{macrocode}
% \end{macro}

% \begin{macro}{\cals@reheight@cells}
% Re-heights all the boxes of a row.
% Retains the widths of these boxes.
%    \begin{macrocode}
\newcommand\cals@reheight@cells[1]{%
\dimen0=\ht#1\relax
\setbox2=\hbox{}%
\def\next{%
 \setbox4=\lastbox
 \ifvoid4
   \def\next{\global\setbox2=\box2}%
 \else
   \dimen4=\wd4
   \setbox4=\vbox to \dimen0{\unvbox4}%
   \ifdim \dimen4=\wd4 \relax \else \wd4=\dimen4 \fi
   \setbox2=\hbox{\box4\unhbox2 }%
 \fi
 \next}%
\setbox0=\hbox{\unhbox#1\next}%
\setbox#1=\box2 }
%    \end{macrocode}
% \end{macro}

% % \subsection{Spanned cells}

% The technical approach:
% \begin{itemize}
% \item Spanning is started in the left top corner using the
%    command |\nullcell{lt...}|
% \item The spanned cell is split on the table cells using the
%    command |\nullcell|. These nullcells are responsible for
%    correct decorations and for calculating the big cell dimensions.
% \item The content of the spanned cells comes in the end,
%    in the right bottom corner.
% \end{itemize}

% It is possible to have several active spans at once, therefore we
% have to remember them. I use a queue. Each time a left column
% of a span is started, we take span data from the queue. After the
% right column of the span, we put the data back to the queue,
% to the end. Probably it is not obvious at the first look (at least,
% I needed time to find this simple idea) that this rotating queue
% is always right: when spanning (its left column) starts again,
% the beginning of the queue always contains data for exactly this spanning.

% \begin{macro}{\cals@spanq@heights}
% The queue for height tracking. In the first version I also had
% a queue for decorations, but later cancelled it.
% I use |\def| instead of |\newcommand| in order that |\show|
% prints "|macro|" instead of "|\long macro|". The latter breaks
% unit tests.
%    \begin{macrocode}
\def\cals@spanq@heights{}
%    \end{macrocode}
% \end{macro}

% \begin{macro}{\cals@span@get}
% Gets |\cals@span@height| from the queue.
%    \begin{macrocode}
\newcommand\cals@span@get{%
\llt@decons\cals@spanq@heights \cals@span@height=\llt@car\relax}
%    \end{macrocode}
% \end{macro}

% \begin{macro}{\cals@span@put}
% Puts |\cals@span@height| to the queue.
%    \begin{macrocode}
\newcommand\cals@span@put{%
\edef\cals@tmp{\the\cals@span@height}\llt@snoc\cals@spanq@heights\cals@tmp}
%    \end{macrocode}
% \end{macro}

% \begin{macro}{\nullcell}
% The big spanned cell is split on the table cells, which are
% identified by |\nullcell|s. The task is to produce correct
% decorations and to track the parameters of the spanned cell.
% \begin{itemize}
% \item Decorations: if defined, the background color is always
% added to the column separation row.
% \item Decorations: the borders are set to 0pt (disabled),
% except for the borders which are requested by
% the parameter of |\nullcell|: |l| for the left border,
% |t| for the top, |r| for the right, |b| for the bottom.
% \item More precisely, the letters in the argument are not to
% specify which decorations to use, but to specify the location
% of the small cell in the big cell. The use for decorations
% is just an useful side-effect.
% \item Action |l|: take the span data from the queue.
% \item Action |r|: update the height of the current span,
% put the data to the queue.
% \item Action |b|: do not put an empty box to the current row.
% Instead, accumulate the width of the current span.
% (Preparation for |\spancontent|.)
% \end{itemize}
%    \begin{macrocode}
\newcommand\nullcell[1]{%
%    \end{macrocode}
% First of all, parse the argument and set the if-commands |\cals@span@ifX|.
% Then get the width of the cell.
%    \begin{macrocode}
\let\cals@span@ifL=\cals@iffalse
\let\cals@span@ifT=\cals@iffalse
\let\cals@span@ifR=\cals@iffalse
\let\cals@span@ifB=\cals@iffalse
\def\next##1{\ifx\relax##1\let\next=\relax \else
 \expandafter\let\csname cals@span@if##1\endcsname=\cals@iftrue \fi
 \next}%
\uppercase{\next #1}\relax
\llt@rot\cals@colwidths \let\cals@nullcell@width=\llt@car
%    \end{macrocode}
% Action "l": update the height, supress the borders, set the rowspan markers.
%    \begin{macrocode}
\cals@span@ifL\iftrue
 \cals@span@ifT\iftrue
   \cals@span@height=0pt %
 \else
   \cals@span@get
   \advance\cals@span@height by \cals@last@row@height\relax
 \fi
 \cals@span@ifB\iftrue \else
  \let\cals@ifInRspan=\cals@iftrue
  \let\cals@ifLastRspanRow=\cals@iffalse
 \fi
 \let\cals@span@borderL=\cals@borderL \let\cals@span@borderT=\cals@borderT
 \let\cals@span@borderR=\cals@borderR \let\cals@span@borderB=\cals@borderB
\fi
%    \end{macrocode}
% Action "r": put the data to the queue, unless in the end of the spanning.
%    \begin{macrocode}
\cals@span@ifR\iftrue
 \cals@span@ifB\iftrue \relax \else \cals@span@put \fi
\fi
%    \end{macrocode}
% Update the current row or calculate the span width
% (in the case of the bottom row).
%    \begin{macrocode}
\cals@span@ifB\iftrue
 \cals@span@ifL\iftrue
  \cals@span@width=\cals@nullcell@width\relax
 \else
  \advance\cals@span@width by \cals@nullcell@width\relax
 \fi
\else
 \setbox\cals@current@row=\hbox{%
   \unhbox\cals@current@row
   \vbox{\hbox to\cals@nullcell@width{}\vfil}}%
\fi
%    \end{macrocode}
% Update decorations
%    \begin{macrocode}
\cals@span@ifL\iftrue \let\cals@borderL=\cals@span@borderL
  \else \def\cals@borderL{0pt}\fi
\cals@span@ifT\iftrue \let\cals@borderT=\cals@span@borderT
  \else \def\cals@borderT{0pt}\fi
\cals@span@ifR\iftrue \let\cals@borderR=\cals@span@borderR
  \else \def\cals@borderR{0pt}\fi
\cals@span@ifB\iftrue \let\cals@borderB=\cals@span@borderB
  \else \def\cals@borderB{0pt}\fi
\cals@decor@next\cals@nullcell@width
\let\cals@borderL=\cals@span@borderL \let\cals@borderR=\cals@span@borderR
\let\cals@borderT=\cals@span@borderT \let\cals@borderB=\cals@span@borderB
}
%    \end{macrocode}
% \end{macro}

% \begin{macro}{\cals@span@width}
% \begin{macro}{\cals@span@height}
% The width of the span cell. The height of the spanned cell (without
% the last row).
%    \begin{macrocode}
\newdimen\cals@span@width
\newdimen\cals@span@height
%    \end{macrocode}
% \end{macro}
% \end{macro}

% \begin{macro}{\cals@ifInRspan}
% Set to |\cals@iftrue| if the current row is a part of a row span.
% Otherwise |\cals@iffalse|.
% \end{macro}

% \begin{macro}{\cals@ifLastRspanRow}
% Set to |\cals@iftrue| if the current row is the last row of
% of a row span. Otherwise |\cals@iffalse|.
% \end{macro}

% \begin{macro}{\cals@updateRspanMarkers}
% Resets the span markers, which are later updated by |\nullcell|
% to the correct state for the current row.
%    \begin{macrocode}
\newcommand\cals@updateRspanMarkers{%
\ifx \empty\cals@spanq@heights
 \let\cals@ifInRspan=\cals@iffalse
\else
 \let\cals@ifInRspan=\cals@iftrue
\fi
\let\cals@ifLastRspanRow=\cals@iftrue}
%    \end{macrocode}
% \end{macro}

% % \subsection{Row dispatcher}

% \begin{macro}{\cals@row@dispatch}
% Depending if the current row has a rowspan cell or not,
% the execution is different.
%    \begin{macrocode}
\newcommand\cals@row@dispatch{%
\ifx b\cals@current@context
 \cals@ifInRspan\iftrue
  \cals@row@dispatch@span
 \else
  \cals@row@dispatch@nospan
 \fi
\else
 \cals@row@dispatch@nospan
\fi}
%    \end{macrocode}
% \end{macro}

% \begin{macro}{\cals@row@dispatch@nospan}
% After a row is typeset in a box, this macro decides what to do next.
% Usually, it should just add decorations and output the row.
% But if a table break is required, it should put the current row
% to backup, typeset the footer, the break, the header and only then 
% the row from the backup. Summary of main parameters:
% \begin{itemize}
% \item rowsep from the last row (|\cals@last@rs|) and the
%   last context (|\cals@last@context|)
% \item current row (|\cals@current@row|), its decorations
%  (|\cals@current@cs|, |\cals@current@rs@above|,
%  |\cals@current@rs@below|) and context (|\cals@current@context|)
% \end{itemize}
%    \begin{macrocode}
\newcommand\cals@row@dispatch@nospan{%
%    \end{macrocode}
% The header and footer rows are always typeset without further
% considerations.
%    \begin{macrocode}
\let\cals@last@context@bak=\cals@last@context
\ifx h\cals@current@context \else
\ifx f\cals@current@context \else
%    \end{macrocode}
% In the body, if a break is required: do it.
%    \begin{macrocode}
\cals@ifbreak\iftrue
 \setbox\cals@backup@row=\box\cals@current@row
 \setbox\cals@backup@cs=\box\cals@current@cs
 \let\cals@backup@rs@above=\cals@current@rs@above
 \let\cals@backup@rs@below=\cals@current@rs@below
 \let\cals@backup@context=\cals@current@context
 \cals@tfoot@tokens
 \lastrule
 \cals@issue@break
 \cals@thead@tokens
 \setbox\cals@current@row=\box\cals@backup@row
 \setbox\cals@current@cs=\box\cals@backup@cs
 \let\cals@current@rs@above=\cals@backup@rs@above
 \let\cals@current@rs@below=\cals@backup@rs@below
 \let\cals@current@context=\cals@backup@context
\fi\fi\fi
%    \end{macrocode}
% Typeset the row. If the width of the row is more than hsize, then
% the issue-code should not fit the row to hsize.
%    \begin{macrocode}
\ifdim\wd\cals@current@row>\hsize\relax
\def\cals@tohsize{}%
\fi
\cals@issue@row
%    \end{macrocode}
% Consider a table such that thead+row1 do not fit to a page
% (see the unit test |regression/test_010_wrongbreak|).
% Without the next code, the following happens:
% thead and row1 are typeset, but the output procedure is not
% executed yet. Therefore, when row2 is ready, we detect that
% a table break is required and create it. Then the output procedure
% moves thead+row1 on the next page. The result:
% thead and row1 on one page, row2 and the rest on the next page
% instead of the whole table on one page.
% Solution: force a run of the output procedure after
% the first row of a table chunk.
%    \begin{macrocode}
\ifx b\cals@last@context
  {\dimen0=\pagetotal\relax
  \advance\dimen0 by \cals@tfoot@height\relax
  \advance\dimen0 by -\pagegoal
  \ifdim\dimen0>0pt\relax
    \vskip\dimen0
    \penalty9999 % with 10000, the output page builder is not called
    \vskip-\dimen0\relax
  \fi
  }%
\fi
}
%    \end{macrocode}
% \end{macro}

% \begin{macro}{\cals@row@dispatch@span}
% The only specific thing to rowspanned rows is that we should not
% allow breaks between the rows in one group. We put these rows
% to one box, and process this big box as a big row.
%    \begin{macrocode}
\newcommand\cals@row@dispatch@span{%
%    \end{macrocode}
% Output the row to the backup box. If the row is the first
% row in the span, let its decorations will be the decorations
% for the future big row. Also, reset the values of leftskip
% and rightskip to avoid adding them twice, once in a individual
% row, and once to the common span box.
%    \begin{macrocode}
\ifvoid\cals@backup@row
 \setbox\cals@backup@row=\vbox{\box\cals@current@row}%
 \setbox\cals@backup@cs=\box\cals@current@cs
 \let\cals@backup@rs@above=\cals@current@rs@above
 \let\cals@backup@context=\cals@last@context
 \cals@backup@leftskip=\leftskip\relax
 \cals@backup@rightskip=\rightskip\relax
 \let\cals@backup@tohsize=\cals@tohsize
 \leftskip=0pt\relax \rightskip=0pt\relax \def\cals@tohsize{}%
\else
 \setbox\cals@backup@row=\vbox{\unvbox\cals@backup@row
  \cals@issue@row}%
\fi
\let\cals@last@rs@below=\cals@current@rs@below
\let\cals@last@context=\cals@current@context
%    \end{macrocode}
% If this is the last row of the span, create the fake big row
% and use the normal dispatcher.
%
%    \begin{macrocode}
\cals@ifLastRspanRow\iftrue
 \setbox\cals@current@row=\box\cals@backup@row
 \setbox\cals@current@cs=\box\cals@backup@cs
 \let\cals@current@rs@above=\cals@backup@rs@above
 \let\cals@last@context=\cals@backup@context
 \leftskip=\cals@backup@leftskip
 \rightskip=\cals@backup@rightskip
 \let\cals@tohsize=\cals@backup@tohsize
 \cals@row@dispatch@nospan
\fi
}
%    \end{macrocode}
% \end{macro}

% \begin{macro}{\cals@backup@row}
% \begin{macro}{\cals@backup@cs}
% Boxes and skips for backup.
%    \begin{macrocode}
\newbox\cals@backup@row
\newbox\cals@backup@cs
\newskip\cals@backup@leftskip
\newskip\cals@backup@rightskip
%    \end{macrocode}
% \end{macro}
% \end{macro}

% To decide on table breaks and row separation decorations,
% we need to trace context.

% \begin{macro}{\cals@current@context}
% The context of the current row. Possible values,
% set as a "|\let|" to a character:
% \begin{itemize}
% \item n: no context, should not happen when the value is required
% \item h: table header
% \item f: table footer
% \item b: table body
% \end{itemize}
% \end{macro}

% \begin{macro}{\cals@last@context}
% The context of the previous row. Possible values, set as a
% "|\let|" to a character:
% \begin{itemize}
% \item n: there is no previous row (not only the start of a table,
%   but also the start of a table chunk)
% \item h, f, b: table header, footer, body
% \item r: a last rule of the table (or its chunk) is just output.
%   This status is used to allow multiple calls to |\lastrule|.
%   Probably the use of |current| instead of |last| is more logical,
%   but using |last| is more safe. Who knows if an user decides to use
%   |\lastrule| somewhere in a middle of a table.
% \end{itemize}
% \end{macro}

% \begin{macro}{\cals@ifbreak}
% Table breaks can be manual or automatic. The first is easy,
% the second is near to impossible if we take into account
% table headers and footer. The following heuristic seems good.
% 
% Check if the current row plus the footer fits to the
% rest of the page. If not, a break is required. This approach
% is based on two assumptions:
% \begin{itemize}
% \item the height of the footer is always the same, and
% \item any body row is larger than the footer.
% \end{itemize}
%
% More precise and technical description: |\cals@ifbreak| decides
% if an automatic table break is required and leaves the
% macro |\cals@iftrue| (yes) or |\cals@iffalse| (no) in the input stream.
% If the user sets |\cals@tbreak@tokens| (using |\tbreak|),
% break is forced. Otherwise, no break is allowed:
% \begin{itemize}
% \item In the header
% \item In the footer
% \item Immediately after the header
% \item At the beginning of a chunk of a table.
% \end{itemize}
% Otherwise break is recommended when
% the sum of the height of the current row and of the footer part
% is greater as the rest height of the page.
% The implicit first parameter is used for if-fi balancing,
% see |\cals@iftrue|.
%    \begin{macrocode}
\newcommand\cals@ifbreak[1]{}
\def\cals@ifbreak{%
\let\cals@tmp=\cals@iffalse
\let\cals@tmpII=\cals@iftrue
\ifx\relax\cals@tbreak@tokens
 \ifx h\cals@current@context \else
  \ifx f\cals@current@context \else
   \ifx h\cals@last@context \else
    \ifx n\cals@last@context \else
      \dimen0=\pagetotal\relax
      \advance\dimen0 by \ht\cals@current@row\relax
      %\showthe\ht\cals@current@row\relax
      \ifx \cals@tfoot@tokens\relax \else
        %\show\cals@tfoot@height\relax
        \advance\dimen0 by \cals@tfoot@height\relax
      \fi
      %\showthe\dimen0\relax
      \ifdim \dimen0>\pagegoal\relax
        \let\cals@tmp=\cals@tmpII
      \fi
 \fi\fi\fi\fi
\else \let\cals@tmp=\cals@tmpII % tbreak@tokens
\fi
\cals@tmp}
%    \end{macrocode}
% \end{macro}

% \begin{macro}{\cals@issue@break}
% By default, force a page break, otherwise use user's tokens
% set by |\tbreak|.
%    \begin{macrocode}
\newcommand\cals@issue@break{\ifx \relax\cals@tbreak@tokens \penalty-10000 %
\else \cals@tbreak@tokens \fi
\let\cals@tbreak@tokens=\relax
\let\cals@last@context=n}
%    \end{macrocode}
% \end{macro}

% \begin{macro}{\cals@set@tohsize}
% \begin{macro}{\cals@tohsize}
% Table row contains not only the row itself, but also |\leftskip|
% and |\rightskip|. Now the dilemma. If the row is just |\hbox|,
% than the glue component is ignored, and the table always aligned
% left. On the other side, if the row is |\hbox to \hsize|, then
% the user gets underfulled boxes. A simple solution is to
% switch on and off the |hsize|-part depending on the skips.
%    \begin{macrocode}
\newcommand\cals@tohsize{}
\newcommand\cals@set@tohsize{\def\cals@tohsize{}%
\ifnum\gluestretchorder\leftskip>0\relax \def\cals@tohsize{to \hsize}\fi
\ifnum\gluestretchorder\rightskip>0\relax \def\cals@tohsize{to \hsize}\fi
}
%    \end{macrocode}
% \end{macro}
% \end{macro}

% \begin{macro}{\cals@activate@rtl}
% \begin{macro}{\cals@deactivate@rtl}
% \begin{macro}{\cals@hbox}
% For bidi support, use |\hboxR| instead of |\hbox|.
% Actually, more is required for bidi support, and
% these macros are retained only as "legacy".
% Otherwise I'd move the code into the beginning of `calstable`.
%    \begin{macrocode}
\newcommand\cals@hbox{}
\newcommand\cals@activate@rtl{\let\cals@hbox=\hboxR}
\newcommand\cals@deactivate@rtl{\let\cals@hbox=\hbox}
\cals@deactivate@rtl
%    \end{macrocode}
% \end{macro}
% \end{macro}
% \end{macro}

% \begin{macro}{\cals@issue@rowsep@alone}
% Typesets the top (or bottom) frame of a table:
% combines |\cals@current@rs@above| and |\cals@framers@width|
% and outputs the row separator.
%    \begin{macrocode}
\newcommand\cals@issue@rowsep@alone{%
\setbox0=\cals@hbox\cals@tohsize{%
 \cals@hskip@lr\leftskip\rightskip
 \cals@rs@sofar@reset
 \cals@rs@joinOne\cals@framers@width\cals@current@rs@above
 \cals@rs@sofar@end
 \cals@hskip@lr\rightskip\leftskip}%
\ht0=0pt \dp0=0pt \box0 }
%    \end{macrocode}
% \end{macro}

% \begin{macro}{\cals@issue@rowsep}
% Combine row separations |\cals@last@rs@below| and |\cals@current@rs@above|,
% taking into considiration the width of the rule:
% \begin{itemize}
% \item n to h, f, b (the top frame): use |\cals@framers@width|
%   and ignore |last@rs@below| because we don't have it
% \item h to h, b to b, f to f (the usual separator): use |\cals@rs@width|
% \item for all other combinations (header to body, body to footer),
% including impossible: use |\cals@bodyrs@width|
% \end{itemize}
%    \begin{macrocode}
\newcommand\cals@issue@rowsep{%
\ifx n\cals@last@context \cals@issue@rowsep@alone \else
 \ifx \cals@last@context\cals@current@context
   \let\cals@tmpIII=\cals@rs@width     \else
   \let\cals@tmpIII=\cals@bodyrs@width \fi
 \setbox0=\cals@hbox\cals@tohsize{%
  \cals@hskip@lr\leftskip\rightskip
  \cals@rs@sofar@reset
  \cals@rs@joinTwo\cals@tmpIII\cals@last@rs@below\cals@current@rs@above
  \cals@rs@sofar@end
  \cals@hskip@lr\rightskip\leftskip}%
 \ht0=0pt \dp0=0pt \box0 %
\fi}
%    \end{macrocode}
% \end{macro}


% \begin{macro}{\cals@last@row@height}
% For spanning support, we need to remember the height of the last row
%    \begin{macrocode}
\newdimen\cals@last@row@height
%    \end{macrocode}
% \end{macro}

% \begin{macro}{\cals@issue@row}
% Typesets the current row and its decorations, then updates
% the last context. Regards |\leftskip| and |\rightskip|
% by putting them inside the row.
%    \begin{macrocode}
\newcommand\cals@issue@row{%
%    \end{macrocode}
% Decorations: first the column separation, then the row separation.
%    \begin{macrocode}
\nointerlineskip
\setbox0=\vtop{\cals@hbox\cals@tohsize{\cals@hskip@lr\leftskip\rightskip
\box\cals@current@cs \cals@hskip@lr\rightskip\leftskip}}%
\ht0=0pt\relax\box0
\nointerlineskip
\cals@issue@rowsep
\nointerlineskip
%    \end{macrocode}
% Output the row, update the last context.
%    \begin{macrocode}
\cals@hbox\cals@tohsize{\cals@hskip@lr\leftskip\rightskip
\box\cals@current@row \cals@hskip@lr\rightskip\leftskip}%
\let\cals@last@rs@below=\cals@current@rs@below
\let\cals@last@context=\cals@current@context
\nobreak}
%    \end{macrocode}
% \end{macro}

% \subsection{Table elements}

% \begin{environment}{\calstable}
% Setup the parameters and let the row dispatcher to do all the work.
%    \begin{macrocode}
\newenvironment{calstable}[1][\cals@table@alignment]{%
\if@RTL\@RTLtabtrue\cals@activate@rtl\fi
\let\cals@thead@tokens=\relax
\let\cals@tfoot@tokens=\relax
\let\cals@tbreak@tokens=\relax
\cals@tfoot@height=0pt \relax
\let\cals@last@context=n%
\let\cals@current@context=b%
\parindent=0pt \relax%
\cals@setup@alignment{#1}%
\cals@setpadding{Ag}\cals@setcellprevdepth{Al}\cals@set@tohsize%
%% Alignment inside is independent on center/flushright outside
\parfillskip=0pt plus1fil\relax
\let\cals@borderL=\relax
\let\cals@borderR=\relax
\let\cals@borderT=\relax
\let\cals@borderB=\relax
\cals@AtBeginTable
}{% End of the table
\cals@tfoot@tokens\lastrule\cals@AtEndTable}
%    \end{macrocode}
% \end{environment}

% \begin{macro}{\cals@AtBeginTable}
% \begin{macro}{\cals@AtEndTable}
% Callbacks for more initialization possibilities.
%    \begin{macrocode}
\newcommand\cals@AtBeginTable{}%
\newcommand\cals@AtEndTable{}%
%    \end{macrocode}
% \end{macro}
% \end{macro}

% \begin{macro}{\lastrule}
% Typesets the last rule (bottom frame) of a table chunk.
% Repeatable calls are ignored.
% Useful in the macro |\tfoot|.
%    \begin{macrocode}
\newcommand\lastrule{%
\ifx r\cals@last@context \relax \else
 \let\cals@last@context=r%
 \nointerlineskip
 \let\cals@current@rs@above=\cals@last@rs@below\cals@issue@rowsep@alone%
\fi}
%    \end{macrocode}
% \end{macro}

% \begin{macro}{\thead}
% Table: the header. Remember for later use, typeset right now.
%    \begin{macrocode}
\newcommand\thead[1]{%
\def\cals@thead@tokens{\let\cals@current@context=h%
#1\let\cals@current@context=b}%
\cals@thead@tokens}
%    \end{macrocode}
% \end{macro}

% \begin{macro}{\tfoot}
% Table: the footer. Remember for later use. Right now, typeset
% to a box to calculate an expected height for the table breaker
% |\cals@ifbreak|.
%    \begin{macrocode}
\newcommand\tfoot[1]{%
\def\cals@tfoot@tokens{\let\cals@current@context=f#1}%
\setbox0=\vbox{\cals@tfoot@tokens}%
\cals@tfoot@height=\ht0 \relax}
%    \end{macrocode}
% \end{macro}

% \begin{macro}{\cals@tfoot@height}
% The height of the footer.
%    \begin{macrocode}
\newdimen\cals@tfoot@height
%    \end{macrocode}
% \end{macro}

% \begin{macro}{\tbreak}
% Table: force a table break. Argument should contain something
% like |\penalty-10000 |.
%    \begin{macrocode}
\newcommand\tbreak[1]{\def\cals@tbreak@tokens{#1}}
%    \end{macrocode}
% \end{macro}

% \begin{macro}{\cals@table@alignment}
% The default alignment of tables in the text flow.
% Doesn't affect the text alignment inside cells.
% \begin{itemize}
% \item |n|: no settings, the default |\leftskip| and |\rightskip| are used
% \item |l|: align left
% \item |c|: align center
% \item |r|: align right
% \end{itemize}
% This setting is appeared in the version 2.3. Earlier versions
% worked as it were |n|.
%    \begin{macrocode}
\newcommand\cals@table@alignment{l}
%    \end{macrocode}
% \end{macro}

% % \subsection{List list of tokens}
%
% Two-dimensional arrays of tokens, or lists of lists of tokens.
%
%
% Format of the list:
%
% \begin{verbatim}
% {...tokens1...}{...tokens2...}...{...tokensN...}
% \end{verbatim}
%
% Token manipulation should not belong to the ``cals'' package,
% and the macros from this section have the prefix |llt@|
% instead of |cals@|. Probably it is better to use some CTAN
% package, but initially the llt-code was small and simple, so
% I did not want dependencies, and now I do not want to replace
% working code with something new.
%
% In comments to these functions, a parameter of type \textit{token list}
% is a macro which will be expanded once to get the tokens, and
% \textit{list list} is a macro which stores the two-dimensional array.
%
% An example of use:
% \begin{verbatim}
% \def\aaa{aaa}
% \def\bbb{bbb}
% \def\ccc{ccc}
% \def\lst{}         % empty list
% \llt@cons\bbb\lst  % \lst -> "{bbb}"
% \llt@snoc\lst\ccc  % \lst -> "{bbb}{ccc}"
% \llt@cons\aaa\lst  % \lst -> "{aaa}{bbb}{ccc}"
% \llt@decons\lst    % \llt@car -> "aaa", \lst -> "{bbb}{ccc}"
% \llt@rot\lst       % \llt@car -> "bbb", \lst -> "{ccc}{bbb}"
% \end{verbatim}
%
%

%
% \begin{macro}{\llt@cons}
% Prepends the token list \#1 to the list list \#2.
% Corrupts the token registers 0 and~2.
%    \begin{macrocode}
\def\llt@cons#1#2{%
\toks0=\expandafter{#1}%
\toks2=\expandafter{#2}%
\edef#2{\noexpand{\the\toks0}\the\toks2 }%
}
%    \end{macrocode}
% \end{macro}

%
% \begin{macro}{\llt@snoc}
% Appends the token list \#2 to the list list \#1 (note the order
% of parameters).
% Macro corrupts the token registers 0 and 2.
%
%    \begin{macrocode}
\def\llt@snoc#1#2{%
\toks0=\expandafter{#1}%
\toks2=\expandafter{#2}%
\edef#1{\the\toks0 \noexpand{\the\toks2}}%
}
%    \end{macrocode}
% \end{macro}

%
% \begin{macro}{\llt@car}
% A token list, set as a side-effect of the list deconstruction
% and rotation functions.
% \end{macro}

%
% \begin{macro}{\llt@decons}
% List deconstruction. The first item is removed from the list list \#1
% and its tokens are put to the token list |\llt@car|.
% Corrupts the token register 0.
% Undefined behaviour if the list list has no items.
%
% The actual work happens on the |\expandafter| line.
% It's hard to explain, let me show the macro expansion,
% I hope it's self-explaining.
%
% \begin{verbatim}
% \expandafter\llt@decons@open\lst}      -->
% \llt@decons@open{aaa}{bbb}{ccc}{ddd}}  -->
% \def\llt@car{aaa} \toks0=\llt@opengroup {bbb}{ccc}{ddd}}  -->
% \def\llt@car{aaa} \toks0={{bbb}{ccc}{ddd}}
% \end{verbatim}
% 
% Why I use |\let\llt@opengroup={| inside the definition?
% Only to balance the number of opening and closing brackets.
% Otherwise TeX will not compile the definition.
%
% Initially I tried to use the following helper:
%
% \begin{verbatim}
% \def\decons@helper#1#2\relax{%
%   \def\llt@car{#1}%
%   \def\list{#2}}
% \end{verbatim}
%
% If a call is |\decons@helper{aaa}{bbb}{ccc}\relax| then
% all is ok, the helper gets: \#1 is |aaa| and \#2 is
% |{bbb}{ccc}|.
%
% Unfortunately, if the list has two items and the call is
% |\decons@helper{aaa}{bbb}}\relax|, then the helper gets:
% \#1 is |aaa| and \#2 is |bbb| instead of |{bbb}|.
% The grouping tokens are lost, and we can't detect it.
%
%    \begin{macrocode}
\def\llt@decons@open#1{%
\def\llt@car{#1}%
\toks0=\llt@opengroup
}

\def\llt@decons#1{%
\let\llt@opengroup={%
\expandafter\llt@decons@open#1}%
\edef#1{\the\toks0}%
}
%    \end{macrocode}
% \end{macro}

%
% \begin{macro}{\llt@rot}
% Rotates the list list \#1. The first item becomes the last. Also,
% its tokens are saved to token list |\llt@car|.
% The second item becomes the first item, the third the second etc.
% Corrupts the token registers 0 and 2.
%
%    \begin{macrocode}
\def\llt@rot#1{%
\ifx#1\empty
\let\llt@car=\relax
\else
\llt@decons#1%
\llt@snoc#1\llt@car%
\fi
}
%    \end{macrocode}
% \end{macro}

%
% \begin{macro}{\llt@desnoc}
% Very unefficient list deconstruction. The last item is removed
% from the list list \#1 % and its tokens are put to the token
% list |\llt@car|. Corrupts the token registers 0 and 2.
% Undefined behaviour if the list list has no items.
%
% |\llt@desnocII| is used to hide |\if| from the loop.
%    \begin{macrocode}
\def\llt@desnocII#1{
\ifx\empty#1%
\let\llt@tmp=n%
\else
\llt@snoc{\llt@newlist}{\llt@car}%
\let\llt@tmp=y%
\fi
}

\def\llt@desnoc#1{%
\def\llt@newlist{}%
\loop
\llt@decons{#1}%
\llt@desnocII{#1}%
\if y\llt@tmp \repeat
\let#1=\llt@newlist}
%    \end{macrocode}
% \end{macro}

% % \section{Decorations}

% How much decoration requires a table? Initially I thought to implement
% a generic approach, so an user could extend the set of what is
% possible---for example, a dashed border instead of a solid one.
% But this is a hard task for me, therefore I switched back
% to the fixed set of properties. Indeed, the use case for cals
% tables is technical documentation, not a wanna-be designer showcases.
% 
% The fixed set of decoration properties:
% \begin{itemize}
% \item padding
% \item border thickness
% \item cancelled after some thinking: border color
% \item cell background
% \end{itemize}
%
% In the first version of this package, decorations could be defined
% for all cells in a row, for all cells in a column and finally
% specially for a cell. Unfortunately, this approach does not work
% for borders of multipage tables, when the decorations on a break are
% different to the internal decorations. Trying different
% workarounds, I finally found a descriptive and direct approach:
% define default decorations for a cell plus define
% decorations for the table frame.
%
% How to decorate the common border of two cells? The following
% seems reasonable:
% \begin{itemize}
% \item The priority of settings: from the user, from the table
%   frame, default. If cells have different priorities, use the
%   highest one.
% \item When priorities are the same, use the maximal thinkness.
% \end{itemize}
%
% Imagine that the user uses only default decorations. Then all
% the internal vertical borders are the same, and all internal
% horizontal border are also the same. It took me time to
% understand this obvious thing, that for the default setup
% we need to define settings only for vertical and horizontal
% borders, not for all four borders.
%
% Column separation and row separation are two very different
% creatures. The former is fixed after a row is processed,
% the latter could change somewhen later due to a table break.

% \subsection{Setters and getters}

% No more setter and getters provided to discourage cell formatting.
% At the moment, if you really need it, use the knowledge of
% the internal variables.

% \begin{macro}{\cals@cs@width}
% Width of column separators (vertical borders). For all the widths, |0pt| disables the rule.
% \begin{macro}{\cals@framecs@width}
% Width of the left and right table frame border.
% \begin{macro}{\cals@rs@width}
% Width of row separators (horizontal borders).
% \begin{macro}{\cals@framers@width}
% Width of the top and bottom table frame border.
% \begin{macro}{\cals@bodyrs@width}
% Width of row separators between the body and the header or the footer.
% \begin{macro}{\cals@bgcolor}
% Background color of the cells. If the macro is empty, it means no background.
%    \begin{macrocode}
\newcommand\cals@cs@width{.4pt}
\newcommand\cals@framecs@width{0pt}
\newcommand\cals@rs@width{.4pt}
\newcommand\cals@framers@width{0pt}
\newcommand\cals@bodyrs@width{1.2pt}
\newcommand\cals@bgcolor{}
%    \end{macrocode}
% \end{macro}
% \end{macro}
% \end{macro}
% \end{macro}
% \end{macro}
% \end{macro}

% \begin{macro}{\cals@borderL}
% \begin{macro}{\cals@borderT}
% \begin{macro}{\cals@borderR}
% \begin{macro}{\cals@borderB}
% Overrides for the widths of the cell borders (left, top, right or bottom).
% Macros, set to |\relax| by the |calstable| environment.
% \end{macro}
% \end{macro}
% \end{macro}
% \end{macro}

% Padding parameters (|\cals@paddingL| etc) and related
% macros (|\alignC| etc) are defined near the macro |\cell|.

% \subsection{Decorations for a row}

% The whole code in this section is devoted to provide functionality
% for the functions |\cals@decor@begin|, |...@next|, |...@end|. After a row is ended,
% we have the following decorations:
% \begin{itemize}
% \item column separation: box |\cals@current@cs|
% \item rowsep specification for the top: macro |\cals@rs@above|
% \item rowsep specification for the bottom: macro |\cals@rs@below|
% \end{itemize}
% The high-level approach is obvious: we construct the decorations
% cell-by-cell. First, we calculate column separation, getting
% the width of the left and the right border. Then we use these
% values to update the border above and below. Unfortunately,
% there is a lot of small details. For example, as explained later,
% we construct the decorations with a delay, therefore the end-function
% is just a special sort of the next-function.

% \begin{macro}{\cals@decor@begin}
% Initialization.
%    \begin{macrocode}
\newcommand\cals@decor@begin{\cals@csrow@begin\cals@rs@spec@begin}
%    \end{macrocode}
% \end{macro}

% \begin{macro}{\cals@decor@next}
% Updates the decorations. The argument is the width of the cell.
%    \begin{macrocode}
\newcommand\cals@decor@next[1]{%
\cals@csrow@nextcell{#1}\cals@borderL\cals@borderR\cals@bgcolor
\cals@rs@spec@next{#1}\cals@lastLeftWidth\cals@borderT\cals@borderB}
%    \end{macrocode}
% \end{macro}

% \begin{macro}{\cals@decor@end}
% Finishes the decorations. Uses |\cals@lastLeftWidth|,
% which is the width of the last right border.
%    \begin{macrocode}
\newcommand\cals@decor@end{%
\cals@csrow@end
\cals@rs@spec@end\cals@lastLeftWidth}
%    \end{macrocode}
% \end{macro}

% \subsection{Deciding on the width and color}

% \begin{macro}{\cals@widthII}
% \begin{macro}{\cals@withWidthII}
% Calculate a final width from the default one (argument 1)
% and user-specified (argument 2, |\relax| means use the default),
% put it to the macro |\cals@width|. Also, the macro |withWidthII|
% start a conditional construction,
% true branch is executed when the width is not zero.
% Unused \#3 is for if-fi balancing, see |\cals@iftrue|.
%    \begin{macrocode}
\newcommand\cals@widthII[2]{%
\ifx \relax#2\edef\cals@width{#1}%
        \else \edef\cals@width{#2}\fi}

\newcommand\cals@withWidthII[3]{%
\cals@widthII{#1}{#2}%
\ifdim \cals@width>0pt }
%    \end{macrocode}
% \end{macro}
% \end{macro}

% \begin{macro}{\cals@withColorII}
% Calculate a final color from the default one (argument 1)
% and user-specified (argument 2, |\relax| means use the default),
% set the macro |\cals@color| to it. The both arguments must
% be macro names, not token lists. Start a conditional
% construction, true branch is executed when the color name is
% given (not empty).
% Unused \#3 is for if-fi balancing, see |\cals@iftrue|.
%    \begin{macrocode}
\newcommand\cals@withColorII[3]{%
\ifx \relax#2\edef\cals@color{#1}%
        \else \edef\cals@color{#2}\fi
%    \end{macrocode}
% Reversing a condition. Based on |\ifnot| macro by David Kastrup
% posted to comp.text.tex 2 March 1998, Message-ID:
% |<m2en0lebuc.fsf@mailhost.neuroinformatik.ruhr-uni-bochum.de>#1/1|.
%    \begin{macrocode}
\ifx \cals@color\empty \else\expandafter\expandafter\fi\iffalse\iftrue\fi}
%    \end{macrocode}
% \end{macro}

% \begin{macro}{\cals@halfWidthToDimen}
% Puts the half of the width in |#2| to the dimension
% register |#1|.
%    \begin{macrocode}
\newcommand\cals@halfWidthToDimen[2]{%
\dimen#1=#2\relax \divide\dimen#1 by 2 }
%    \end{macrocode}
% \end{macro}

% \begin{macro}{\cals@maxWidth}
% Of two widths, given as macros, selects the maximal and put
% the result to |\cals@width|. Takes care for |\relax|.
%    \begin{macrocode}
\newcommand\cals@maxWidth[2]{%
\ifx \relax#1\relax
  \ifx \relax#2\let\cals@width=\relax
          \else \edef\cals@width{#2}\fi
\else
  \ifx \relax#2\relax
    \edef\cals@width{#1}%
  \else
    \ifdim #1>#2 \edef\cals@width{#1}%
           \else \edef\cals@width{#2}\fi\fi\fi}
%    \end{macrocode}
% \end{macro}

% \begin{macro}{\cals@iftrue}
% \begin{macro}{\cals@iffalse}
% Balancing if-fi. The following does not work:
% \begin{verbatim}
% \let\next=\iftrue ...
% \if ... \next ... \fi ... \fi
% \end{verbatim}
% But this code does work:
% \begin{verbatim}
% \let\next=\cals@iftrue ...
% \if ... \next\iftrue ... \fi ... \fi
% \end{verbatim}
% We use |\iftrue| (or any other if-start), which is taken into
% account when scanning for the fi-else-if balance,
% but ignored when executed.
%    \begin{macrocode}
\def\cals@iftrue#1{\iftrue}
\def\cals@iffalse#1{\iffalse}
%    \end{macrocode}
% \end{macro}
% \end{macro}

% % \subsection{Column separation (colsep) and cell background}

% In-row decorations are the vertical borders between the cells
% and also the background color of the cells.

% \begin{macro}{\cals@cs@outOne}
% Typesets the background and the left border of a cell.
% Decorations have zero depth and undefined height.
% Parameters:
% \begin{enumerate}
% \item Width of the cell. The use of |\relax| avoids typesetting
%   the cell itself, which is used when creating the right frame
%   of a table.
% \item Width of the border. 0pt is no border.
% \item Color of the background. Empty macro is no color.
% \end{enumerate}
% If some arguments are undefined (through |\relax|), global
% variables are used: |\cals@bgcolor| and |\cals@cs@width|.
%
% Corrupts |dimen0|.
%    \begin{macrocode}
\newcommand\cals@cs@outOne[3]{%
%    \end{macrocode}
% Create the full-width background
%    \begin{macrocode}
\ifx \relax#1%
\else
 \cals@withColorII\cals@bgcolor{#3}\iftrue
  \textcolor{\cals@color}{\vrule depth0pt width#1 }%
  \hskip -#1\relax
 \fi
\fi
%    \end{macrocode}
% The border. I feel that overprinting the background is good,
% but I don't know why I think so.
%    \begin{macrocode}
\cals@withWidthII\cals@cs@width{#2}\iftrue
 \cals@halfWidthToDimen0 \cals@width %
 \hskip -\dimen0 %
 \vrule depth0pt width\cals@width\relax
 \hskip -\dimen0 %
\fi
%    \end{macrocode}
% We will need the actual width of the left border in
% a grand-grand-...-caller, when constructing a rowsep specification.
%    \begin{macrocode}
\let\cals@lastLeftWidth=\cals@width
%    \end{macrocode}
% Add width to skip to the next cell.
%    \begin{macrocode}
\ifx \relax#1\else \hskip#1 \fi
}
%    \end{macrocode}
% \end{macro}

% \begin{macro}{\cals@current@cs}
% The box to store column separation.
%    \begin{macrocode}
\newbox\cals@current@cs
%    \end{macrocode}
% \end{macro}

% \begin{macro}{\cals@csrow@begin}
% \begin{macro}{\cals@csrow@nextcell}
% \begin{macro}{\cals@csrow@end}
% Constructs an hbox with colsep decorations. Call to the begin-macro
% re-initializes |\cals@current@cs| and makes that the left table frame border
% is of the correct width. The end-macro creates the right border for
% the right table frame. The most work is performed in the nextcell-macro.
% Arguments:
% \begin{enumerate}
% \item Width of the cell
% \item Width of the left border
% \item Width of the right border, \emph{must be a macro name}
% \item Background color of the cell
% \end{enumerate}
% For the special conditions (|\relax|, 0pt, empty name) see
% the description of |\cals@cs@outOne|.
% The right border is not typeset immediately. Instead, it is saved
% to |\cals@lastWidth| (as |\relax| if no overrides)
% and is handled by the next call to |nextcell|.
%    \begin{macrocode}
\newcommand\cals@csrow@begin{%
\setbox\cals@current@cs=\box\voidb@x %
\let\cals@lastWidth=\relax}

\newcommand\cals@csrow@nextcell[4]{%
%    \end{macrocode}
% For the first cell, temporarily re-define the default border width
% to the frame border. Macro |\next| will restore it back.
%    \begin{macrocode}
\ifvoid\cals@current@cs
  \toks0=\expandafter{\cals@cs@width}%
  \def\next{\edef\cals@cs@width{\the\toks0}}%
  \edef\cals@cs@width{\cals@framecs@width}%
\else \let\next=\relax \fi
%    \end{macrocode}
% Create the decorations, remember the right border.
% Restore the value |lastLeftWidth| after the end of the hbox-group.
%    \begin{macrocode}
\cals@maxWidth\cals@lastWidth{#2}%
\setbox\cals@current@cs=\hbox{\unhbox\cals@current@cs
    \cals@cs@outOne{#1}\cals@width{#4}%
    \global\let\cals@tmp=\cals@lastLeftWidth}%
\let\cals@lastLeftWidth=\cals@tmp
\let\cals@lastWidth=#3%
%    \end{macrocode}
% Restore the old value of the default border width.
%    \begin{macrocode}
\next}

%    \end{macrocode}
% User-specified width (initially by |borderR|, now located in
% |lastWidth|) must override the frame width.
%    \begin{macrocode}
\newcommand\cals@csrow@end{%
\ifx \relax\cals@lastWidth
 \let\cals@width=\cals@framecs@width
\else
 \let\cals@width=\cals@lastWidth
\fi
\cals@csrow@nextcell\relax\cals@width\relax\relax}
%    \end{macrocode}
% \end{macro}
% \end{macro}
% \end{macro}

% % \subsection{Row separation (rowsep)}

% \subsubsection{Data presentation}

% A horizontal line between table rows can't appear in the output
% immediately. Its formatting should be postponed until the next
% row and its context are known. Two basic cases:
% \begin{itemize}
% \item The default cell formatting is to have a border around cells.
% For some cell \textit{A} in the middle of a table, the user has
% overridden formatting to no borders. To imlement the user's wish,
% we should not create the bottom border for the cell \textit{B}
% which is above the \textit{A}. But we don't know about the wish
% for \textit{A} while processing \textit{B}. Therefore, the border
% between two rows can be established only after processing the
% both rows.
% \item The default cell border is 1pt, but a border between
% a body and a header or a footer row is 2pt. It means that the
% border between two rows should be created only after we sure
% that there is no table break.
% \end{itemize}
% Our approach is to define the desired formatting of a rowsep
% as a set of parameters in a token list. Later we can join two
% rowseps or a rowsep and context to a final rowsep.
%
% A rowsep token list consist of several items, each
% item is a list of tokens or token groups:
% \begin{enumerate}
% \item length
% \item left-border
% \item right-border
% \item user-specified thickness
% \end{enumerate}
% Values \textit{left-border} and \textit{right-border} are
% required to get a nice rectangle border around a cell. Without length
% correction, using the cell dimension, the border could look like:
% \begin{verbatim}
%   xxxxxxxx
% xxx cell xxx
%   xxxxxxxx
% \end{verbatim}
% With length correction, we get the correct result:
% \begin{verbatim}
% xxxxxxxxxxxx
% xxx cell xxx
% xxxxxxxxxxxx
% \end{verbatim}
%
% Here is an example of a rowsep specification. It consist of three items.
% The first item: length is 5cm, width 2pt, borders are 2mm.
% The second item: length is 9cm, borders are 2mm, no rule at all.
% The third item: length 2cm, default width, borders are 2pt.
% The token list for this specification:
% \begin{verbatim}
% { {5cm} {2mm} {2mm} {2pt} }
% { {9cm} {2mm} {2mm} {0pt} }
% { {2cm} {2pt} {2pt} \relax }
% \end{verbatim}
%
% Cell length and the left and right borders should be the correct lengths,
% the rule thickness can be |\relax|, in this case the actual thickness
% will be calculated during output.
%
% \begin{macro}{\cals@rs@pack}
% Construct a rowsep fragment from the arguments 2-5 and
% put it to the macro 1.
%    \begin{macrocode}
\newcommand\cals@rs@pack[5]{%
\edef#1{\noexpand{#2\noexpand}\noexpand{#3\noexpand}\noexpand{#4\noexpand}%
 \ifx \relax#5\relax \else \noexpand{#5\noexpand}\fi }}
%    \end{macrocode}
% \end{macro}

% \begin{macro}{\cals@rs@unpack}
% The reverse for |\cals@rs@pack|. The first argument is a rowsep
% fragment (without enclosing curly braces), arguments 2-5 is macro
% names where to put the results.
%    \begin{macrocode}
\newcommand\cals@rs@unpack[5]{%
\def\cals@tmp##1##2##3##4{\edef#2{##1}\edef#3{##2}\edef#4{##3}%
 \ifx\relax##4\let#5=\relax \else \edef#5{##4}\fi}%
\expandafter\cals@tmp#1}
%    \end{macrocode}
% \end{macro}

%
% ^^A -----------------------------------------------------
%
% \subsubsection{From individual decorations to rowsep specification}

% The rowsep specifications are created cell-by-cell and stored
% in the macros |\cals@current@rs@above| and |\cals@current@rs@below|.
% The construction happens with delay because we don't know
% the exact value of the right border until the next cell is processed.

% \begin{macro}{\cals@rs@spec@begin}
% Initializes |\cals@current@rs@above|, |\cals@current@rs@below|
% and set the flag of a new row.
%    \begin{macrocode}
\newcommand\cals@rs@spec@begin{%
\def\cals@current@rs@above{}%
\def\cals@current@rs@below{}%
\let\cals@rs@spec@ll=\relax}
%    \end{macrocode}
% \end{macro}

% \begin{macro}{\cals@rs@spec@next}
% \begin{macro}{\cals@rs@spec@nextII}
% Finalizes the decorations for the previous cell by using the left border
% of the current as the right border for the previous. Then
% remembers the decorations of the current cell ---
% the left border width,
% the widths of the top and bottom borders (|\relax| is ok) ---
% in the macros |\cals@rs@spec@ll|, |...@bl|, |...@bt|, |...@bb|.
% All the arguments much be macros.
%    \begin{macrocode}
\newcommand\cals@rs@spec@next[4]{
\cals@rs@spec@nextII#2
\let\cals@rs@spec@ll=#1%
\let\cals@rs@spec@bl=#2%
\let\cals@rs@spec@bt=#3%
\let\cals@rs@spec@bb=#4%
}

\newcommand\cals@rs@spec@nextII[1]{%
\ifx \relax\cals@rs@spec@ll \else
 \cals@rs@pack\cals@tmp\cals@rs@spec@ll\cals@rs@spec@bl#1\cals@rs@spec@bt
 \llt@snoc\cals@current@rs@above\cals@tmp
 \cals@rs@pack\cals@tmp\cals@rs@spec@ll\cals@rs@spec@bl#1\cals@rs@spec@bb
 \llt@snoc\cals@current@rs@below\cals@tmp
\fi
}
%    \end{macrocode}
% \end{macro}
% \end{macro}

% \begin{macro}{\cals@rs@spec@end}
% Finishes the rowsep specification by putting the last cell to it.
% The only implicit argument (|\cals@lastLeftWidth|) is the width
% of the right border of the last cell.
%    \begin{macrocode}
\newcommand\cals@rs@spec@end[1]{}
\let\cals@rs@spec@end=\cals@rs@spec@nextII
%    \end{macrocode}
% \end{macro}

%
% ^^A -----------------------------------------------------
%
% \subsubsection{``Waiting'' rowsep}
%
% \begin{macro}{\cals@rs@sofar@length}
% \begin{macro}{\cals@rs@sofar@borderl}
% \begin{macro}{\cals@rs@sofar@borderr}
% \begin{macro}{\cals@rs@sofar@width}
% Typesetting a row separator is not an easy task, especially
% because we support border-widths. Indeed, consider the worst
% case: four cells and all the borders are different.
% Our solution is an optimizer for a good case. We do not typeset
% a fragment of the rule immediately. Instead, we remember the
% parameters. If the next fragment is of the same width,
% we increase the length of the ``waiting'' fragment.
% Otherwise, we output the waiting fragment and the new fragment
% becomes the new waiting fragment.
%    \begin{macrocode}
\newcommand\cals@rs@sofar@length{}
\newcommand\cals@rs@sofar@borderl{}
\newcommand\cals@rs@sofar@borderr{}
\newcommand\cals@rs@sofar@width{}
%    \end{macrocode}
% \end{macro}
% \end{macro}
% \end{macro}
% \end{macro}

% \begin{macro}{\cals@rs@sofar@reset}
% Sets a flag that a new waiting rule should be started.
%    \begin{macrocode}
\newcommand\cals@rs@sofar@reset{\let\cals@rs@sofar@width=\relax}
%    \end{macrocode}
% \end{macro}

% \begin{macro}{\cals@rs@sofar@end}
% Prints the waiting rule, if exists.
%    \begin{macrocode}
\newcommand\cals@rs@sofar@end{\ifx\relax\cals@rs@sofar@width
  \else\cals@rs@sofar@out\fi}
%    \end{macrocode}
% \end{macro}

% \begin{macro}{\cals@rs@sofar@next}
% Enlarges the current waiting rule, or typesets it and
% starts new if the widths do not match. All the parameters
% must be macro names. In order: length, left border, right
% border, width.
%    \begin{macrocode}
\newcommand\cals@rs@sofar@next[4]{%
\ifx\relax\cals@rs@sofar@width
%    \end{macrocode}
% Starts a new waiting rule.
%    \begin{macrocode}
 \let\cals@rs@sofar@length=#1%
 \let\cals@rs@sofar@borderl=#2%
 \let\cals@rs@sofar@borderr=#3%
 \let\cals@rs@sofar@width=#4%
\else
 \ifdim \cals@rs@sofar@width=#4\relax
%    \end{macrocode}
% Enlarges the waiting rule.
%    \begin{macrocode}
  \dimen0=\cals@rs@sofar@length\relax
  \advance\dimen0 by #1\relax
  \edef\cals@rs@sofar@length{\the\dimen0}%
  \let\cals@rs@sofar@borderr=#3%
 \else
%    \end{macrocode}
% Typesets the current and start a new waiting rule.
%    \begin{macrocode}
  \cals@rs@sofar@out
  \let\cals@rs@sofar@length=#1%
  \let\cals@rs@sofar@borderl=#2%
  \let\cals@rs@sofar@borderr=#3%
  \let\cals@rs@sofar@width=#4%
 \fi
\fi}
%    \end{macrocode}
% \end{macro}

% \begin{macro}{\cals@rs@sofar@over}
% Repeats the last rowsep fragment, probably with another settings.
% Arguments are like in |\cals@rs@sofar@next|.
%    \begin{macrocode}
\newcommand\cals@rs@sofar@over[4]{%
\ifdim 0pt=#4
  \relax
\else
  \ifdim \cals@rs@sofar@width=#4\relax
%    \end{macrocode}
% The width is not changed. We probably need to enlarge the right border
% and probably the left border too. The latter is a bit harder because
% we don't want to change it if the line continues from another cell
% (so, change only if |length|+|borderl|\textgreater |sofar@length|+|sofar@borderl|).
%    \begin{macrocode}
    \ifdim #3>\cals@rs@sofar@borderr\relax
      \edef\cals@rs@sofar@borderr{#3}%
    \fi
    \dimen0=\cals@rs@sofar@length
    \advance\dimen0 by \cals@rs@sofar@borderl\relax
    \advance\dimen0 by -#2\relax
    \ifdim #1>\dimen0 \relax
      \edef\cals@rs@sofar@borderl{#2}%
    \fi
  \else
%    \end{macrocode}
% Typesets the current and start a new waiting rule.
%    \begin{macrocode}
    \cals@rs@sofar@out
    \hskip-#1\relax
    \let\cals@rs@sofar@length=#1%
    \let\cals@rs@sofar@borderl=#2%
    \let\cals@rs@sofar@borderr=#3%
    \let\cals@rs@sofar@width=#4%
  \fi
\fi}
%    \end{macrocode}
% \end{macro}

% \begin{macro}{\cals@rs@sofar@out}
% Typesets the waiting rule
%    \begin{macrocode}
\newcommand\cals@rs@sofar@out{%
\ifdim 0pt=\cals@rs@sofar@width\relax
  \hskip\cals@rs@sofar@length\relax
\else
  \cals@halfWidthToDimen0\cals@rs@sofar@borderl
  \hskip-\dimen0\relax
  \cals@halfWidthToDimen2\cals@rs@sofar@borderr
  \dimen4=\cals@rs@sofar@length\relax
  \advance\dimen4 by \dimen0\relax \advance\dimen4 by \dimen2\relax
  \cals@halfWidthToDimen6\cals@rs@sofar@width
  \vrule height\dimen6 depth\dimen6 width\dimen4\relax
  \hskip-\dimen2\relax
\fi}
%    \end{macrocode}
% \end{macro}

% ^^A -----------------------------------------------------
%
% \subsubsection{From rowsep specification to typesetting}

% \begin{macro}{\cals@rs@joinTwo}
% Join and typeset two rowseps (arguments 2 and 3, must be macro names).
% The number and the lengths of the fragments in the rowseps
% should match.
% The argument 1 (also a macro name) is the default width.
% Corrupts the macros 2 and 3.
% Call this macro inside |sofar@reset|...|@end| group.
%    \begin{macrocode}
\newcommand\cals@rs@joinTwo[3]{%
%    \end{macrocode}
% The loop function.
%    \begin{macrocode}
\def\next##1{%
\ifx \eol##1\let\next=\relax
\else
  \toks0=\expandafter{##1}%
  \edef\cals@tmpII{\the\toks0}%
  \llt@decons#3%
%    \end{macrocode}
% Now |\cals@tmpII| contains a current fragment of the first rowsep,
% and |\llt@car| of the second. Unpack the individual parameters.
%    \begin{macrocode}
  \cals@rs@unpack\cals@tmpII\cals@tmpLI \cals@tmpBlI \cals@tmpBrI \cals@tmpWI
  \cals@rs@unpack\llt@car   \cals@tmpLII\cals@tmpBlII\cals@tmpBrII\cals@tmpWII
%    \end{macrocode}
% The special case is when we should not typeset a rowsep fragment.
%    \begin{macrocode}
  \let\cals@tmp=\cals@iftrue
  \cals@maxWidth\cals@tmpWI\cals@tmpWII
  \ifx \relax\cals@width\else \ifdim \cals@width=0pt %
    \cals@rs@sofar@next\cals@tmpLI\cals@tmpBlI\cals@tmpBrI\cals@width
    \let\cals@tmp=\cals@iffalse
  \fi\fi
%    \end{macrocode}
% Not the special case. Put the both definitions, and let the
% underlying functions take care of calculations.
%    \begin{macrocode}
  \cals@tmp\ifvoid
    \cals@widthII#1\cals@tmpWI
    \cals@rs@sofar@next\cals@tmpLI\cals@tmpBlI\cals@tmpBrI\cals@width
    \cals@widthII#1\cals@tmpWII
    \cals@rs@sofar@over\cals@tmpLII\cals@tmpBlII\cals@tmpBrII\cals@width
  \fi
\fi
%    \end{macrocode}
% End of |\next| definition: continue the loop.
% End of |\cals@rs@joinTwo| definition: start the loop.
%    \begin{macrocode}
\next}%
\expandafter\next#2\eol}
%    \end{macrocode}
% \end{macro}

% \begin{macro}{\cals@rs@joinOne}
% A simplified version of the previous macro. We have only one rowsep,
% and want to join and typeset it with regard to some width,
% given as the first macro parameter. Call this macro inside
% |sofar@reset|...|@end| group.
%    \begin{macrocode}
\newcommand\cals@rs@joinOne[2]{%
\def\next##1{\ifx\eol##1\let\next=\relax\else
 \toks0=\expandafter{##1}%
 \edef\cals@tmpII{\the\toks0}%
 \cals@rs@unpack\cals@tmpII\cals@tmpL\cals@tmpBl\cals@tmpBr\cals@tmpW
 \cals@widthII#1\cals@tmpW
 \cals@rs@sofar@next\cals@tmpL\cals@tmpBl\cals@tmpBr\cals@width
\fi\next}%
\expandafter\next#2\eol}
%    \end{macrocode}
% \end{macro}

% % \subsection{RTL (right-to-left) hooks}

% \begin{macro}{\if@RTL}
% \begin{macro}{\if@RTLtab}
% \begin{macro}{\@RTLtabtrue}
% Provide RTL status commands even if the RTL packages are not loaded.
%    \begin{macrocode}
\def\next{%
  \let\if@RTL=\iffalse
  \let\if@RTLtab=\iffalse
  \let\@RTLtabtrue=\relax
}
\ifdefined\if@RTL \relax \else \next \fi
%    \end{macrocode}
% \end{macro}
% \end{macro}
% \end{macro}

% \begin{macro}{\cals@setup@alignment}
% TODO
% The default alignment of tables in the text flow.
% Doesn't affect the text alignment inside cells.
% \begin{itemize}
% \item |n|: no settings, the default |\leftskip| and |\rightskip| are used
% \item |l|: align left
% \item |c|: align center
% \item |r|: align right
% \end{itemize}
% This setting is appeared in the version 2.3. Earlier versions
% worked as it were |n|. The subtle difference to |l| is explained
% in comments to |\cals@setup@alignment|.
%    \begin{macrocode}
\newcommand\cals@setup@alignment[1]{%
\ifx c#1 \cals@vfillAdd \leftskip \cals@vfillAdd \rightskip \fi
\if@RTL
 \ifx l#1 \cals@vfillAdd \leftskip \cals@vfillDrop\rightskip \fi
 \ifx r#1 \cals@vfillDrop\leftskip \cals@vfillDrop\rightskip \fi
\else
 \ifx l#1 \cals@vfillDrop\leftskip \cals@vfillDrop\rightskip \fi
 \ifx r#1 \cals@vfillAdd \leftskip \cals@vfillDrop\rightskip \fi
\fi
}
%    \end{macrocode}
% \end{macro}


%
% \Finale
\endinput
