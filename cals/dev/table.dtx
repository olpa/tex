% \subsection{Row dispatcher}

% \begin{macro}{\cals@row@dispatch}
% Depending if the current row has a rowspan cell or not,
% the execution is different.
%    \begin{macrocode}
\newcommand\cals@row@dispatch{%
\ifx b\cals@current@context
 \cals@ifInRspan\iftrue
  \cals@row@dispatch@span
 \else
  \cals@row@dispatch@nospan
 \fi
\else
 \cals@row@dispatch@nospan
\fi}
%    \end{macrocode}
% \end{macro}

% \begin{macro}{\cals@row@dispatch@nospan}
% After a row is typeset in a box, this macro decides what to do next.
% Usually, it should just add decorations and output the row.
% But if a table break is required, it should put the current row
% to backup, typeset the footer, the break, the header and only then 
% the row from the backup. Summary of main parameters:
% \begin{itemize}
% \item rowsep from the last row (|\cals@last@rs|) and the
%   last context (|\cals@last@context|)
% \item current row (|\cals@current@row|), its decorations
%  (|\cals@current@cs|, |\cals@current@rs@above|,
%  |\cals@current@rs@below|) and context (|\cals@current@context|)
% \end{itemize}
%    \begin{macrocode}
\newcommand\cals@row@dispatch@nospan{%
%    \end{macrocode}
% The header and footer rows are always typeset without further
% considerations.
%    \begin{macrocode}
\let\cals@last@context@bak=\cals@last@context
\ifx h\cals@current@context \else
\ifx f\cals@current@context \else
%    \end{macrocode}
% In the body, if a break is required: do it.
%    \begin{macrocode}
\cals@ifbreak\iftrue
 \setbox\cals@backup@row=\box\cals@current@row
 \setbox\cals@backup@cs=\box\cals@current@cs
 \let\cals@backup@rs@above=\cals@current@rs@above
 \let\cals@backup@rs@below=\cals@current@rs@below
 \let\cals@backup@context=\cals@current@context
 \cals@tfoot@tokens
 \lastrule
 \cals@issue@break
 \cals@thead@tokens
 \setbox\cals@current@row=\box\cals@backup@row
 \setbox\cals@current@cs=\box\cals@backup@cs
 \let\cals@current@rs@above=\cals@backup@rs@above
 \let\cals@current@rs@below=\cals@backup@rs@below
 \let\cals@current@context=\cals@backup@context
\fi\fi\fi
%    \end{macrocode}
% Typeset the row.
%    \begin{macrocode}
\cals@issue@row
%    \end{macrocode}
% Consider a table such that thead+row1 do not fit to a page
% (see the unit test |regression/test_010_wrongbreak|).
% Without the next code, the following happens:
% thead and row1 are typeset, but the output procedure is not
% executed yet. Therefore, when row2 is ready, we detect that
% a table break is required and create it. Then the output procedure
% moves thead+row1 on the next page. The result:
% thead and row1 on one page, row2 and the rest on the next page
% instead of the whole table on one page.
% Solution: force a run of the output procedure after
% the first row of a table chunk.
%    \begin{macrocode}
\ifx b\cals@last@context
 \ifx h\cals@last@context@bak \vskip0pt \penalty10000 \fi
 \ifx n\cals@last@context@bak \vskip0pt \penalty10000 \fi
\fi
}
%    \end{macrocode}
% \end{macro}

% \begin{macro}{\cals@row@dispatch@span}
% The only specific thing to rowspanned rows is that we should not
% allow breaks between the rows in one group. We put these rows
% to one box, and process this big box as a big row.
%    \begin{macrocode}
\newcommand\cals@row@dispatch@span{%
%    \end{macrocode}
% Output the row to the backup box. If the row is the first
% row in the span, let its decorations will be the decorations
% for the future big row. Also, reset the values of leftskip
% and rightskip to avoid adding them twice, once in a individual
% row, and once to the common span box.
%    \begin{macrocode}
\ifvoid\cals@backup@row
 \setbox\cals@backup@row=\vbox{\box\cals@current@row}%
 \setbox\cals@backup@cs=\box\cals@current@cs
 \let\cals@backup@rs@above=\cals@current@rs@above
 \let\cals@backup@context=\cals@last@context
 \cals@backup@leftskip=\leftskip\relax
 \cals@backup@rightskip=\rightskip\relax
 \let\cals@backup@tohsize=\cals@tohsize
 \leftskip=0pt\relax \rightskip=0pt\relax \def\cals@tohsize{}%
\else
 \setbox\cals@backup@row=\vbox{\unvbox\cals@backup@row
  \cals@issue@row}%
\fi
\let\cals@last@rs@below=\cals@current@rs@below
\let\cals@last@context=\cals@current@context
%    \end{macrocode}
% If this is the last row of the span, create the fake big row
% and use the normal dispatcher.
%
%    \begin{macrocode}
\cals@ifLastRspanRow\iftrue
 \setbox\cals@current@row=\box\cals@backup@row
 \setbox\cals@current@cs=\box\cals@backup@cs
 \let\cals@current@rs@above=\cals@backup@rs@above
 \let\cals@last@context=\cals@backup@context
 \leftskip=\cals@backup@leftskip
 \rightskip=\cals@backup@rightskip
 \let\cals@tohsize=\cals@backup@tohsize
 \cals@row@dispatch@nospan
\fi
}
%    \end{macrocode}
% \end{macro}

% \begin{macro}{\cals@backup@row}
% \begin{macro}{\cals@backup@cs}
% Boxes and skips for backup.
%    \begin{macrocode}
\newbox\cals@backup@row
\newbox\cals@backup@cs
\newskip\cals@backup@leftskip
\newskip\cals@backup@rightskip
%    \end{macrocode}
% \end{macro}
% \end{macro}

% To decide on table breaks and row separation decorations,
% we need to trace context.

% \begin{macro}{\cals@current@context}
% The context of the current row. Possible values,
% set as a "|\let|" to a character:
% \begin{itemize}
% \item n: no context, should not happen when the value is required
% \item h: table header
% \item f: table footer
% \item b: table body
% \end{itemize}
% \end{macro}

% \begin{macro}{\cals@last@context}
% The context of the previous row. Possible values, set as a
% "|\let|" to a character:
% \begin{itemize}
% \item n: there is no previous row (not only the start of a table,
%   but also the start of a table chunk)
% \item h, f, b: table header, footer, body
% \item r: a last rule of the table (or its chunk) is just output.
%   This status is used to allow multiple calls to |\lastrule|.
%   Probably the use of |current| instead of |last| is more logical,
%   but using |last| is more safe. Who knows if an user decides to use
%   |\lastrule| somewhere in a middle of a table.
% \end{itemize}
% \end{macro}

% \begin{macro}{\cals@ifbreak}
% Table breaks can be manual or automatic. The first is easy,
% the second is near to impossible if we take into account
% table headers and footer. The following heuristic seems good.
% 
% Check if the current row plus the footer fits to the
% rest of the page. If not, a break is required. This approach
% is based on two assumptions:
% \begin{itemize}
% \item the height of the footer is always the same, and
% \item any body row is larger than the footer.
% \end{itemize}
%
% More precise and technical description: |\cals@ifbreak| decides
% if an automatic table break is required and leaves the
% macro |\cals@iftrue| (yes) or |\cals@iffalse| (no) in the input stream.
% If the user sets |\cals@tbreak@tokens| (using |\tbreak|),
% break is forced. Otherwise, no break is allowed:
% \begin{itemize}
% \item In the header
% \item In the footer
% \item Immediately after the header
% \item At the beginning of a chunk of a table.
% \end{itemize}
% Otherwise break is recommended when
% the sum of the height of the current row and of the footer part
% is greater as the rest height of the page.
% The implicit first parameter is used for if-fi balancing,
% see |\cals@iftrue|.
%    \begin{macrocode}
\newcommand\cals@ifbreak[1]{}
\def\cals@ifbreak{%
\let\cals@tmp=\cals@iffalse
\let\cals@tmpII=\cals@iftrue
\ifx\relax\cals@tbreak@tokens
 \ifx h\cals@current@context \else
  \ifx f\cals@current@context \else
   \ifx h\cals@last@context \else
    \ifx n\cals@last@context \else
      \dimen0=\pagetotal\relax
      \advance\dimen0 by \ht\cals@current@row\relax
      %\showthe\ht\cals@current@row\relax
      \ifx \cals@tfoot@tokens\relax \else
        %\show\cals@tfoot@height\relax
        \advance\dimen0 by \cals@tfoot@height\relax
      \fi
      %\showthe\dimen0\relax
      \ifdim \dimen0>\pagegoal\relax
        \let\cals@tmp=\cals@tmpII
      \fi
 \fi\fi\fi\fi
\else \let\cals@tmp=\cals@tmpII % tbreak@tokens
\fi
\cals@tmp}
%    \end{macrocode}
% \end{macro}

% \begin{macro}{\cals@issue@break}
% By default, force a page break, otherwise use user's tokens
% set by |\tbreak|.
%    \begin{macrocode}
\newcommand\cals@issue@break{\ifx \relax\cals@tbreak@tokens \penalty-10000 %
\else \cals@tbreak@tokens \fi
\let\cals@tbreak@tokens=\relax
\let\cals@last@context=n}
%    \end{macrocode}
% \end{macro}

% \begin{macro}{\cals@set@tohsize}
% \begin{macro}{\cals@tohsize}
% Table row contains not only the row itself, but also |\leftskip|
% and |\rightskip|. Now the dilemma. If the row is just |\hbox|,
% than the glue component is ignored, and the table always aligned
% left. On the other side, if the row is |\hbox to \hsize|, then
% the user gets underfulled boxes. A simple solution is to
% switch on and off the |hsize|-part depending on the skips.
%    \begin{macrocode}
\newcommand\cals@tohsize{}
\newcommand\cals@set@tohsize{\def\cals@tohsize{}%
\ifnum\gluestretchorder\leftskip>0\relax \def\cals@tohsize{to \hsize}\fi
\ifnum\gluestretchorder\rightskip>0\relax \def\cals@tohsize{to \hsize}\fi
}
%    \end{macrocode}
% \end{macro}
% \end{macro}

% \begin{macro}{\cals@issue@rowsep@alone}
% Typesets the top (or bottom) frame of a table:
% combines |\cals@current@rs@above| and |\cals@framers@width|
% and outputs the row separator.
%    \begin{macrocode}
\newcommand\cals@issue@rowsep@alone{%
\setbox0=\hbox\cals@tohsize{%
 \hskip\leftskip
 \cals@rs@sofar@reset
 \cals@rs@joinOne\cals@framers@width\cals@current@rs@above
 \cals@rs@sofar@end
 \hskip\rightskip}%
\ht0=0pt \dp0=0pt \box0 }
%    \end{macrocode}
% \end{macro}

% \begin{macro}{\cals@issue@rowsep}
% Combine row separations |\cals@last@rs@below| and |\cals@current@rs@above|,
% taking into considiration the width of the rule:
% \begin{itemize}
% \item n to h, f, b (the top frame): use |\cals@framers@width|
%   and ignore |last@rs@below| because we don't have it
% \item h to h, b to b, f to f (the usual separator): use |\cals@rs@width|
% \item for all other combinations (header to body, body to footer),
% including impossible: use |\cals@bodyrs@width|
% \end{itemize}
%    \begin{macrocode}
\newcommand\cals@issue@rowsep{%
\ifx n\cals@last@context \cals@issue@rowsep@alone \else
 \ifx \cals@last@context\cals@current@context
   \let\cals@tmpIII=\cals@rs@width     \else
   \let\cals@tmpIII=\cals@bodyrs@width \fi
 \setbox0=\hbox\cals@tohsize{%
  \hskip\leftskip
  \cals@rs@sofar@reset
  \cals@rs@joinTwo\cals@tmpIII\cals@last@rs@below\cals@current@rs@above
  \cals@rs@sofar@end
  \hskip\rightskip}%
 \ht0=0pt \dp0=0pt \box0 %
\fi}
%    \end{macrocode}
% \end{macro}


% \begin{macro}{\cals@last@row@height}
% For spanning support, we need to remember the height of the last row
%    \begin{macrocode}
\newdimen\cals@last@row@height
%    \end{macrocode}
% \end{macro}

% \begin{macro}{\cals@issue@row}
% Typesets the current row and its decorations, then updates
% the last context. Regards |\leftskip| and |\rightskip|
% by putting them inside the row.
%    \begin{macrocode}
\newcommand\cals@issue@row{%
%    \end{macrocode}
% Decorations: first the column separation, then the row separation.
%    \begin{macrocode}
\nointerlineskip
\setbox0=\vtop{\hbox\cals@tohsize{\hskip\leftskip \box\cals@current@cs \hskip\rightskip}}%
\ht0=0pt\relax\box0
\nointerlineskip
\cals@issue@rowsep
\nointerlineskip
%    \end{macrocode}
% Output the row, update the last context.
%    \begin{macrocode}
\hbox\cals@tohsize{\hskip\leftskip \box\cals@current@row \hskip\rightskip}%
\let\cals@last@rs@below=\cals@current@rs@below
\let\cals@last@context=\cals@current@context}
%    \end{macrocode}
% \end{macro}

% \subsection{Table elements}

% \begin{environment}{\calstable}
% Setup the parameters and let the row dispatcher to do all the work.
%    \begin{macrocode}
\newenvironment{calstable}[1][\cals@table@alignment]{%
\if@RTL\@RTLtabtrue\fi
\let\cals@thead@tokens=\relax
\let\cals@tfoot@tokens=\relax
\let\cals@tbreak@tokens=\relax
\cals@tfoot@height=0pt \relax
\let\cals@last@context=n%
\let\cals@current@context=b%
\parindent=0pt \relax%
\cals@setup@alignment{#1}%
\cals@setpadding{Ag}\cals@setcellprevdepth{Al}\cals@set@tohsize%
%% Alignment inside is independent on center/flushright outside
\parfillskip=0pt plus1fil\relax
\let\cals@borderL=\relax
\let\cals@borderR=\relax
\let\cals@borderT=\relax
\let\cals@borderB=\relax
\cals@AtBeginTable
}{% End of the table
\cals@tfoot@tokens\lastrule\cals@AtEndTable}
%    \end{macrocode}
% \end{environment}

% \begin{macro}{\cals@AtBeginTable}
% \begin{macro}{\cals@AtEndTable}
% Callbacks for more initialization possibilities.
%    \begin{macrocode}
\newcommand\cals@AtBeginTable{}%
\newcommand\cals@AtEndTable{}%
%    \end{macrocode}
% \end{macro}
% \end{macro}

% \begin{macro}{\lastrule}
% Typesets the last rule (bottom frame) of a table chunk.
% Repeatable calls are ignored.
% Useful in the macro |\tfoot|.
%    \begin{macrocode}
\newcommand\lastrule{%
\ifx r\cals@last@context \relax \else
 \let\cals@last@context=r%
 \nointerlineskip
 \let\cals@current@rs@above=\cals@last@rs@below\cals@issue@rowsep@alone%
\fi}
%    \end{macrocode}
% \end{macro}

% \begin{macro}{\thead}
% Table: the header. Remember for later use, typeset right now.
%    \begin{macrocode}
\newcommand\thead[1]{%
\def\cals@thead@tokens{\let\cals@current@context=h%
#1\let\cals@current@context=b}%
\cals@thead@tokens}
%    \end{macrocode}
% \end{macro}

% \begin{macro}{\tfoot}
% Table: the footer. Remember for later use. Right now, typeset
% to a box to calculate an expected height for the table breaker
% |\cals@ifbreak|.
%    \begin{macrocode}
\newcommand\tfoot[1]{%
\def\cals@tfoot@tokens{\let\cals@current@context=f#1}%
\setbox0=\vbox{\cals@tfoot@tokens}%
\cals@tfoot@height=\ht0 \relax}
%    \end{macrocode}
% \end{macro}

% \begin{macro}{\cals@tfoot@height}
% The height of the footer.
%    \begin{macrocode}
\newdimen\cals@tfoot@height
%    \end{macrocode}
% \end{macro}

% \begin{macro}{\tbreak}
% Table: force a table break. Argument should contain something
% like |\penalty-10000 |.
%    \begin{macrocode}
\newcommand\tbreak[1]{\def\cals@tbreak@tokens{#1}}
%    \end{macrocode}
% \end{macro}

% \begin{macro}{\cals@table@alignment}
% The default alignment of tables in the text flow.
% Doesn't affect the text alignment inside cells.
% \begin{itemize}
% \item |n|: no settings, the default |\leftskip| and |\rightskip| are used
% \item |l|: align left
% \item |c|: align center
% \item |r|: align right
% \end{itemize}
% This setting is appeared in the version 2.3. Earlier versions
% worked as it were |n|. The subtle difference to |l| is explained
% in comments to |\cals@setup@alignment|.
%    \begin{macrocode}
\newcommand\cals@table@alignment{l}
%    \end{macrocode}
% \end{macro}
